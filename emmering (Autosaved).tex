% Options for packages loaded elsewhere
\PassOptionsToPackage{unicode}{hyperref}
\PassOptionsToPackage{hyphens}{url}
%
\documentclass[
]{article}
\usepackage{lmodern}
\usepackage{amssymb,amsmath}
\usepackage{ifxetex,ifluatex}
\ifnum 0\ifxetex 1\fi\ifluatex 1\fi=0 % if pdftex
  \usepackage[T1]{fontenc}
  \usepackage[utf8]{inputenc}
  \usepackage{textcomp} % provide euro and other symbols
\else % if luatex or xetex
  \usepackage{unicode-math}
  \defaultfontfeatures{Scale=MatchLowercase}
  \defaultfontfeatures[\rmfamily]{Ligatures=TeX,Scale=1}
\fi
% Use upquote if available, for straight quotes in verbatim environments
\IfFileExists{upquote.sty}{\usepackage{upquote}}{}
\IfFileExists{microtype.sty}{% use microtype if available
  \usepackage[]{microtype}
  \UseMicrotypeSet[protrusion]{basicmath} % disable protrusion for tt fonts
}{}
\makeatletter
\@ifundefined{KOMAClassName}{% if non-KOMA class
  \IfFileExists{parskip.sty}{%
    \usepackage{parskip}
  }{% else
    \setlength{\parindent}{0pt}
    \setlength{\parskip}{6pt plus 2pt minus 1pt}}
}{% if KOMA class
  \KOMAoptions{parskip=half}}
\makeatother
\usepackage{xcolor}
\IfFileExists{xurl.sty}{\usepackage{xurl}}{} % add URL line breaks if available
\IfFileExists{bookmark.sty}{\usepackage{bookmark}}{\usepackage{hyperref}}
\hypersetup{
  hidelinks,
  pdfcreator={LaTeX via pandoc}}
\urlstyle{same} % disable monospaced font for URLs
\usepackage{graphicx}
\makeatletter
\def\maxwidth{\ifdim\Gin@nat@width>\linewidth\linewidth\else\Gin@nat@width\fi}
\def\maxheight{\ifdim\Gin@nat@height>\textheight\textheight\else\Gin@nat@height\fi}
\makeatother
% Scale images if necessary, so that they will not overflow the page
% margins by default, and it is still possible to overwrite the defaults
% using explicit options in \includegraphics[width, height, ...]{}
\setkeys{Gin}{width=\maxwidth,height=\maxheight,keepaspectratio}
% Set default figure placement to htbp
\makeatletter
\def\fps@figure{htbp}
\makeatother
\setlength{\emergencystretch}{3em} % prevent overfull lines
\providecommand{\tightlist}{%
  \setlength{\itemsep}{0pt}\setlength{\parskip}{0pt}}
\setcounter{secnumdepth}{-\maxdimen} % remove section numbering

\author{}
\date{}

\begin{document}

\includegraphics[width=2.86111in,height=3.21528in]{media/image1.jpeg}

\textbf{NATIONAL INSTITUTE OF TECHNOLOGY RAIPUR}

\textbf{Submitted to}:

Dr. Saurabh

\textbf{Submitted by}:

Name : Varsha Singh

\textbf{Branch}: Biomedical Engineering.

\textbf{E-mail}: singhvarsha1526@gmail.com

\textbf{Emerging technologies in health care} :-

Digital health today effects every business in the health care space
whether it's hospital , insurance companies , life science companies ,
public health agencies , multilateral agencies , family foundation ,
non-profits , national health systems. They are all being affected by
digital health because ever things that we are doing as society today is
being digitized. By this program is so important is that health care has
been one of the most resistant industries to digitization over the last
20 year's.

What we are seeing now really is almost a catch up and that's where this
program will really help executives accelerate there understanding of
how to do this Not just from identification standpoint of what are the
most health digital available today? But also a static standpoint how
does digital work for our institution how to do roll out our digital
health studies within our institution.

Emerging Technology of Healthcare

1. Virtual Concierge

A virtual concierge can perform many of the tasks in-person, from
answering emails and phone calls and scheduling appointments and events
to responding to complaints and requests and even handling payments. Our
front door solution is an excellent example of the many applications
that virtual concierge has. Their flagship software is an AI care
navigation assistant that helps patients find a clinic or physician,
answers questions, schedules appointments, and even screens for COVID-19
symptoms. Healthcare facilities benefit from enhanced remote access
capabilities, improved patient satisfaction, and better healthcare
utilization.~

2. Artificial Intelligence

With the potential to radically transform
healthcare,~\href{https://gyant.com/products/}{artificial
intelligence}~can help professionals make better judgments and reduce
human error and the risk of preventable scenarios. From radiology tools
and immunotherapy for cancer patients to identifying infectious disease
patterns, advanced technology helps develop more efficient and precise
interventions. As learning algorithms evolve and become more accurate,
they are likely to significantly impact healthcare services, including
diagnostic approaches, treatments, and care processes.~~

3. Voice Search

Voice search technology is becoming increasingly popular globally, with
a~\href{https://www.npr.org/about-npr/577007267/jan-2018-smart-audio-report}{survey}~by
Edison Search and NRP showing that nearly 17 percent of Americans now
use a voice-activated speaker. Smart technology assistants are also
changing the way patients search for care, including locating a hospital
and asking about illness symptoms. The 2019 Voice Assistant Consumer
Adoption in
Healthcare~\href{https://voicebot.ai/wp-content/uploads/2019/10/voice_assistant_consumer_adoption_in_healthcare_report_voicebot.pdf}{report}~reveals
that 51.9 percent of consumers are prepared to use voice-search
solutions, and 7.5 percent have already done so. Participants mainly use
assistants to inquire about symptoms (72.9 percent), ask about medical
information (45.9 percent), locate urgent care, clinic, or hospital
(37.7 percent), and research treatment options (37.7 percent).~

4. Virtual Reality

Also an emerging technology in healthcare, virtual reality (VR) is an
innovative tool with many applications, from teaching autistic children
communication and social skills to engaging patients in activities and
games for rehabilitation purposes.~

VR solutions can manage hot flashes through cognitive behavioral therapy
and relieve pain through meditation training. Some applications use
Google Glass and augmented reality for clinical and medical
documentation support, including reminder, order, and referral
assistance.~

Based on HIPAA compliant cloud infrastructure, VR solutions help
healthcare providers to create patient summaries and notes, answer
physician requests, pend orders and create referral letters. Virtual
reality can be used across various settings such as surgery centers,
hospitals, emergency rooms, home visits, medical offices, urgent care
clinics, pop-up clinics, and telemedicine.~

5. Mobile Apps

Relatively new technology in healthcare, mobile applications can upload
patients' medical records, check in, schedule appointments, and provide
expert advice. A host of different mobile apps now helps physicians with
patient monitoring and management, information gathering, consulting,
and health record access and maintenance.~

Applications also assist healthcare professionals with medical training
and education, clinical decision-making, and time and data management.
They come in handy in providing physicians with communication tools such
as email, text, video conferencing, and voice calling. Physicians also
need access to laboratory information systems, picture archiving and
communication systems, and electronic medical and health records.
Healthcare professionals use clinical software applications such as
medical calculators and disease diagnoses tools.

\end{document}
